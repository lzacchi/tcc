\chapter{Conclusões}

Os trabalhos relacionados escolhidos apresentam um panorama inicial bastante informativo, quando analizados em conjunto:
\begin{itemize}
    \item A relevância de missões CubeSat cresce a cada ano, observada pelo crescente volume de lançamentos;
    \item São projetos com potencial educacional elevado, sendo usados por insituições de ensino para capacitar e treinar estudantes e também desenvolver, testar novas tecnologias e processos de desenvolvimento.
\end{itemize}
 
 Os dois artigos elaborados pela equipe do FloripaSat-1 demonstram a enorme quantidade de conhecimento gerada, bem como a experiência adquirida em projeto e desenvolvimento de tecnologias espaciais.

Percebeu-se também que, embora exista uma biblioteca ampla sobre testes de hardware de \textit{CubeSats}, a quantidade de material distinto que foque apenas (ou principalmente) em teste de software espacial é comparativamente menor. Espera-se que esse trabalho seja, então, uma boa fonte de informação sobre o potencial que esta área pode apresentar para estudos e inovação.

O estudo destes trabalhos relacionados evidência também a importância de um projeto de testes estruturado. Mais de uma vez, nos artigos selecionados anteriormente foi descrito que a etapa de testes foi importante para a descoberta e correção de erros de design, sejam eles de harware ou software. É nessa vertente que este trabalho propõe uma nova aproximação para os testes de software, por meio de um sistema de workflows que permita a execução de testes de maneira automática, fazendo uso da \textit{FlatSat} da missão FloripaSat-2 para testar de maneira simultânea, os sistemas embarcados de todos os módulos do satélite, bem como as interações e comunicações entre eles.

Se bem sucedido, esse modelo potencializará a descoberta e correção de erros e possibiliatará o desenvolvimento de programas mais confiáveis e de forma mais rápida.


\section{Trabalhos Futuros}
Como sugestão e planejamento de trabalhos futuros, sobretudo dando continuidade à este relatório durante a disciplina de Trabalho de Conclusão de Curso II, sugere-se alguns pontos de aprofundamento:
\begin{itemize}
    \item Aprofundar a pesquisa sobre testes de software, trazendo uma breve análise do estado da arte;
    \item Trazer mais trabalhos relacionados sobre testes de software embarcado, e software para aplicações espaciais;
    \item Implementar e apresentar o sistema de \textit{workflows} à ser utilizado em conjunto com a FlatSat do FLoripaSat-2;
    \item Armazenar, preservar, e analisar os registros e histórico de execuções dos fluxos de testes automatizados;
    \item Redigir um estudo de caso trazendo as principais informações e conclusões, tendo como base os dados coletados.
\end{itemize}