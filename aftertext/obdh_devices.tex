% ----------------------------------------------------------
\chapter{\textit{Workflow} de testes de unidade para o módulo OBDH 2.0}
% ----------------------------------------------------------
\label{cod:obdh_devices}
Este capítulo apresenta os códigos necessários para a execução do \textit{workflow} de automação de testes de unidade do módulo OBDH 2.0 \cite{obdh2-github} do FloripSat-2

\begin{section}{unit-tests-devices.yml}
\label{unit-tests-devices.yml}
    \begin{minted}
    [
    frame=lines,
    framesep=2mm,
    baselinestretch=1.2,
    breaklines=True,
    fontsize=\footnotesize,
    linenos
    ]
    {yaml}

#
# unit-tests-devices.yml
#
# Copyright (C) 2021, SpaceLab.
#
# This file is part of OBDH 2.0.
#
# OBDH 2.0 is free software: you can redistribute it and/or modify
# it under the terms of the GNU General Public License as published by
# the Free Software Foundation, either version 3 of the License, or
# (at your option) any later version.
#
# OBDH 2.0 is distributed in the hope that it will be useful,
# but WITHOUT ANY WARRANTY; without even the implied warranty of
# MERCHANTABILITY or FITNESS FOR A PARTICULAR PURPOSE. See the
# GNU General Public License for more details.
#
# You should have received a copy of the GNU General Public License
# along with OBDH 2.0. If not, see <http://www.gnu.org/licenses/>.
#
#


name: Devices Unit Tests

on:
  push:
    branches: [ dev_firmware ]
  pull_request:
    branches: [ master, dev, dev_firmware]

  # 'workflow_dispatch' allows manual execution
  # of this workflow under the repository's 'Actions' tab
  workflow_dispatch:

jobs:

  # Generates Matrix
  # This job executes the 'deployJSON.py' script
  # which compiles a list of all files ending in '_test.c'
  # in a given directory and includes them in a .json file
  # along with the path to the executable file.
  generate-matrix:
    name:

    runs-on: ubuntu-latest

    outputs:
      matrix: ${{ steps.set-matrix.outputs.matrix }}

    steps:
      # Checks-out the repository under $GITHUB_WORKSPACE, so the job can navigate it
      - uses: actions/checkout@v2

      - name: Create JSON file
        run: python3 .github/workflows/deployJSON.py --source firmware/tests/devices/

      - name: Resulting JSON file for matrix generation
        run: echo "$(cat .github/workflows/test-list.json)"

      # Set the matrix output from the JSON (manipulated to remove spaces and replace \n -> %0A, " -> \")
      - id: set-matrix
        name: Set matrix output from the JSON file
        run: echo "::set-output name=matrix::$( echo "$(cat .github/workflows/test-list.json)" | sed ':a;N;$!ba;s/\n/%0A/g' )"

  # This job reads the matrix containing the paths
  # created by the previous job and runs each program
  # individually, i.e. spawning one job for every
  # test file included in the matrix.
  run-tests:
    name: run-tests
    needs: generate-matrix
    runs-on: ubuntu-latest

    strategy:
      fail-fast: false

      matrix: ${{fromJson(needs.generate-matrix.outputs.matrix)}}


    env:
      MAKE_TARGET: ${{ matrix.name }}
      TEST_FILE: ${{ matrix.test_name }}
      TEST_PATH: ${{ matrix.path }}

    steps:
      - uses: actions/checkout@v2
      # Install required libs
      - name: Install CMocka
        run: sudo apt-get install libcmocka0 libcmocka-dev

      - name: Signal make target
        run: echo "Generating make file for $MAKE_TARGET"

      - name: Generate Test File
        run: cd firmware/tests/devices && make $MAKE_TARGET

      - name: Signal running test
        run: echo "Running $TEST_FILE"

      - name: Run test
        run: $TEST_PATH

    \end{minted}
\end{section}

\begin{section}{deployJSON.py}
\label{deployJSON.py}
    \begin{minted}
    [
    frame=lines,
    framesep=2mm,
    baselinestretch=1.2,
    breaklines=True,
    fontsize=\footnotesize,
    linenos
    ]
    {yaml}
    #
# deployJSON.py
#
# Copyright (C) 2021, SpaceLab.
#
# This file is part of OBDH 2.0.
#
# OBDH 2.0 is free software: you can redistribute it and/or modify
# it under the terms of the GNU General Public License as published by
# the Free Software Foundation, either version 3 of the License, or
# (at your option) any later version.
#
# OBDH 2.0 is distributed in the hope that it will be useful,
# but WITHOUT ANY WARRANTY; without even the implied warranty of
# MERCHANTABILITY or FITNESS FOR A PARTICULAR PURPOSE. See the
# GNU General Public License for more details.
#
# You should have received a copy of the GNU General Public License
# along with OBDH 2.0. If not, see <http://www.gnu.org/licenses/>.
#
#

import sys
import os
import json

if len(sys.argv) <= 2:
    print("\nWrong arguments")
    print("Use: python3 test-deployer --source <target directory>\n")

    sys.exit(1)
else:
    if sys.argv[1] == '--source':
        # Directory path and list
        path = sys.argv[2]
        if path.startswith('/'):
            path = '.' + path
        elif not path.startswith('./'):
            path = './' + path
        file_dir = os.listdir(path)

    files = {
        "include": []
    }

    # Naming convention:
    #   - test files should end with '_test.c';
    #   - and executables should be named '<target>_unit_test
    #   - makefile commands should be 'make <target>' for each test file
    #
    # This script should work for all folders that follow these guidelines
    # otherwise this script will have to be edited to function properly

    for file in file_dir:
        if file.endswith('_test.c'):
            file_name = file.replace('.c', '')
            test_file = file_name.replace('_test', '_unit_test')
            file_path = path + test_file

            file_info = {
                "name": file_name,
                "test_name": test_file,
                "path": file_path
            }
            files['include'].append(file_info)

    # Convert the dictionary to JSON format
    json_target = json.dumps(files)

    with open(".github/workflows/test-list.json", "w") as json_file:
        json_file.write(json_target)
    json_file.close()

    print("JSON created successfully for folder " + sys.argv[2])
    \end{minted}
\end{section}
\newpage


